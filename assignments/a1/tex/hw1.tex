% --------------------------------------------------------------
% This is all preamble stuff that you don't have to worry about.
% Head down to where it says "Start here"
% --------------------------------------------------------------

\documentclass[12pt]{article}


\usepackage[margin=1in]{geometry} 
\usepackage{graphicx}
\usepackage{epstopdf}
\usepackage{amsmath,amsthm,amssymb}
\usepackage[procnames]{listings}
\usepackage{float}
\usepackage{color}

\newcommand{\N}{\mathbb{N}}
\newcommand{\Z}{\mathbb{Z}}

\newenvironment{theorem}[2][Theorem]{\begin{trivlist}
\item[\hskip \labelsep {\bfseries #1}\hskip \labelsep {\bfseries #2.}]}{\end{trivlist}}
\newenvironment{lemma}[2][Lemma]{\begin{trivlist}
\item[\hskip \labelsep {\bfseries #1}\hskip \labelsep {\bfseries #2.}]}{\end{trivlist}}
\newenvironment{exercise}[2][Exercise]{\begin{trivlist}
\item[\hskip \labelsep {\bfseries #1}\hskip \labelsep {\bfseries #2.}]}{\end{trivlist}}
\newenvironment{problem}[2][Problem]{\begin{trivlist}
\item[\hskip \labelsep {\bfseries #1}\hskip \labelsep {\bfseries #2.}]}{\end{trivlist}}
\newenvironment{question}[2][Question]{\begin{trivlist}
\item[\hskip \labelsep {\bfseries #1}\hskip \labelsep {\bfseries #2.}]}{\end{trivlist}}
\newenvironment{corollary}[2][Corollary]{\begin{trivlist}
\item[\hskip \labelsep {\bfseries #1}\hskip \labelsep {\bfseries #2.}]}{\end{trivlist}}

\DeclareMathOperator*{\argmin}{arg\,min}
\DeclareMathOperator*{\argmax}{arg\,max}

\begin{document}

% --------------------------------------------------------------
%                         Start here
% --------------------------------------------------------------

\title{Assignment 1}%replace X with the appropriate number
\author{Kelvin Xu\\ %replace with your name
COMP652} %if necessary, replace with your course title

\maketitle
\begin{problem}{1. (a)}
See function load\_data
\end{problem}

\begin{problem}{1. (b)}
With each datacase as a row
\end{problem}
\begin{align*}
    Xw &= Y 
\end{align*}
By doing this, we can see that we are not changing any of the decision
boundaries. We are just are only effecting the regression coefficients, and
their corresponding intepretation. This approach also has the added advantage
that it can help avoid numerical instabilities.

Finally, normalization of this type, and other similar normalization schemes
helps make the model invariant to parameterization. For example, the ridge
regression classifier will have very different weights for the same data
shifted around because the penalty will shrink the weight matrix
often at the expense of accuracy. 

\begin{problem}{1. (c)}
See function polyRegress 
\end{problem}

\begin{problem}{1. (d)}
See function cross\_valid\_regress 
\end{problem}

\begin{problem}{1. (e)}
\end{problem}

\begin{problem}{1. (f)}
\end{problem}

\begin{problem}{2.}
Writing out the log-likelihood:
\end{problem}
\[
    L = \prod_{i=1}^{m} P(y_i \mid x_i ; h ) P(x_i)
\]
We are interested in the argmax of this value, we 
also take a log which doesn't change the argmax since it 
is a monotonically increasing function.
\begin{align*}
    \argmax_w L &= \argmax \log l \\
    l &= \sum_{i=1}^{m} \log P(y_i \mid x_i ; w ) + \log P(x_i)
\end{align*}
Discarding the $P(x_i)$ term since it does not depend on w.
\begin{align*}
    l &= \sum_{i=1}^{m} \log P(y_i \mid x_i ; w ) \\
      &= \sum_{i=1}^{m} \log \Bigg( \frac{1}{\sqrt{2\pi\sigma_i^2}} \exp \Big(-\frac{1}{2} \frac{(y_i - h_w(x_i))^2}{\sigma_i^2} \Big) \Bigg) \\ 
      &= \sum_{i=1}^{m} \log \Big( \frac{1}{\sqrt{2\pi\sigma_i^2}} \Big) - \sum_{i=1}^{m} \Big(\frac{1}{2} \frac{(y_i - h_w(x_i))^2}{\sigma_i^2} \Big)
\end{align*}
Dropping the first term since it has no dependence on w
\begin{align*}
    l  &= -\sum_{i=1}^{m} \Big(\frac{1}{2} \frac{(y_i - h_w(x_i))^2}{\sigma_i^2} \Big)
\end{align*}
We can see here that maximizing this quantity is equivalent to minimizing
\begin{align*}
    l  &= \sum_{i=1}^{m} \Big(\frac{1}{2} \frac{(y_i - h_w(x_i))^2}{\sigma_i^2} \Big)
\end{align*}
Taking a derivative with respect to W gives:
\begin{align*}
    \frac{\partial l}{\partial w} &= \sum_{i=1}^{m} \Big(\frac{1}{2} \frac{(y_i - h_w(x_i))^2}{\sigma_i^2} \Big)
                                  &= \sum_{i=1}^{m} \Big(\frac{(y_i - h_w(x_i))^2}{\sigma_i^2} \Big)
\end{align*}

\begin{problem}{3. (a)}
\end{problem}

\begin{problem}{3. (b)}
\end{problem}

\begin{problem}{3. (c)}
\end{problem}

\begin{problem}{4.}
\end{problem}
% --------------------------------------------------------------
%     You don't have to mess with anything below this line.
% --------------------------------------------------------------

\end{document}
